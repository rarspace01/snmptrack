\chapter{Einleitung}
\label{cha:Einleitung}

Zielsetzung\\
\\
Der Verfasser war während der 5. Praxisphase in der IT-Abteilung im Bereich Infrastruktur eingesetzt.
Zu den Aufgaben der Infrastruktur gehört die Sicherstellung des Betriebes der Anwendungssoftware. Um den Einsatz dieser Software zu ermöglichen kommt eine Vielzahl von Hardware zum Einsatz. Ein Teil der Hardware stellt unter anderem das Netzwerk dar. Neben dem Aufbau und der Konfiguration des Netzwerkes zählt auch der Betrieb. Um den Betrieb des Netzwerks sicherzustellen ist es notwendig ein Überblick über dieses zu behalten.
Im Netzwerkmanagement gibt es hierfür einen speziellen Part, welcher sich unter dem Begriff des Netzwerkmonitoring zusammenfassen lässt. Für dieses Umfeld gibt es diverse Software um diese Aufgabe zu bewerkstelligen. Je nach Software können sowohl Hardware als auch Anwendungen kontrolliert werden. Diese Arbeit konzentriert sich vor allem auf die Erfassung der Geräte in einem Netzwerk. Aufgrund der Tatsache, dass bei Pirelli Deutschland GmbH eine Aktualisierung der kostenpflichtigen Lösung ‘CiscoWorks’ ansteht, soll aus Kostengründen ein Entwurf eines System zur Erfassung aller Geräte im Netzwerk (‘Usertracking’) entwickelt werden.
In der Arbeit soll ein System entworfen werden, welches auf der Grundlage der Anforderungen von dem Unternehmen, sowie aufgrund eigener Untersuchungen, basiert. In der Arbeit selbst wird nicht im Detail auf die implementierte Lösung eingegangen, vielmehr wird sich auf den Entwurf und dessen Entscheidungsfindung konzentriert, da diese das Grundgerüst für die Implementierung liefern. Somit entfallen auch Erläuterungen zu eingesetzten Algorithmen und dem Quellcode.
Im Mittelpunkt soll das Vorgehen, hilfreiche Konzepte aus der Theorie und eine kritische Betrachtung der Ausgangslage stehen, um eine optimale Umsetzung zu garantieren.\\
\\
Motivation\\
\\
Zunächst ist ist es vor allem interessant die bestehenden Kosten zu senken. Vor allem im Zug des immer weiter verbreiteten Einsatzes von OpenSource-Software im gewerblichen Bereich muss untersucht werden, ob es für die bisher Eingesetzte Lösung ein Ersatz gibt bzw. ob eine eigene Lösung mit der Verwendung bereits existierender OpenSource-Software bewerkstelligt werden kann.
Neben der monetären Aspekte spielen aber auch technische Argumente eine Rolle.
So können bei eigenen Entwicklungen beispielsweise Anpassungen vorgenommen werden, die bei Standardsoftware nicht möglich sind. Z.b. besteht dann die Möglichkeit Switchs von anderen Herstellern ebenfalls auszulesen, aber auch neuere Geräte ohne größere Verzögerung von dem System zu unterstützen.
\\
\\Vorgehensweise\\
\\
Hierzu soll zunächst auf wichtige Grundlagen eingegangen werden, welche zu einer Implementierung notwendig sind. Im Anschluss soll eine Untersuchung der Ausgangslage erfolgen und anhand dieser die entsprechenden Anforderungen abgeleitet und diese wiederum kritisch untersucht werden. Im Anschluss werden bereits existierende Lösungen untersucht und deren Eigenschaften erläutert. Danach wird die Entscheidung für eine eigene Implementierung eines Systems getroffen. Um dies bewerkestelligen zu können wird ein Entwurf eines solchen Systemes gemacht. Daraufhin werden eine Vielzahl von Untersuchungen angestellt, um die Designentscheidungen im Bezug auf deren Auswirkung auf die Umsetzung der Anforderungen sicherzustellen. In diesem Zusammenhang wird auch auf den Zeitplan, sowie die Probleme bei der Implementierung eingegangen werden.
Zum Ende der Arbeit werden zusätzlich mögliche Anwendungsfelder der Implementierten Lösung angesprochen und eine Wirtschaftliche Betrachtung durchgeführt, welche überprüfen soll, ob die Implementierung eines eigenen Softwaresystems eine ökonomisch sinnvolle Entscheidung ist.\\