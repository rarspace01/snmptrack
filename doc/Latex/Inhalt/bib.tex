\chapter*{Quellen/Literaturverzeichnis}
\addcontentsline{toc}{chapter}{Quellen/Literaturverzeichnis}

%\begin{thebibliography}{------}


%\bibitem[CODD1]{codd1}
%	,E. F. Codd.%
%    {\em A Relational Model of Data for Large Shared Data Banks}.
%	Commun. ACM, 1970, S. 377-387
Chen, P. (1976): The Entity-Relationship Model--Toward a Unified View of Data. In: ACM Transactions on Database Systems, Vol 1, No 1
Burnus, H. (2007): Datenbankentwicklung in IT-Berufen, 1. Auflage
Codd, E. F. (1970): A Relational Model of Data for Large Shared Data Banks in Commun. ACM, Vol 13, Nr. 6,
Alpar, P. (2001): Vorlesung, Datenorganisation und Datenbanken,  http://www.tekinci.de/skripte/DBDM/DB-SS2001.pdf,
Schubert, M. (2007): Datenbanken, Theorie, Entwurf und Programmierung relationaler Datenbanken, 2. Auflage
Falkowski, B. (2002): Business Computing: Grundlagen und Standardsoftware, 1. Auflage
Singh, S. (2009):Database Systems: Concepts, Design \& Applications, 
Tiobe Software, TIOBE Programming Community Index for January 2011
Schönsleben, P. (2007): Integrales Logistikmanagement. Planung und Steuerung der umfassenden Supply Chain, Paul Schönsleben, , Verlag: Springer, Berlin; Auflage: 4., überarb. u. erw. A. (März 2007)
Haas P./Johner, C. (2009), Praxishandbuch IT im Gesundheitswesen. Erfolgreich einführen, entwickeln, anwenden und betreiben, Christian Johner, , Peter Haas, Hanser Fachbuch (5. März 2009)
Schwabe, G./Streitz, N./Unland, R. (2001) CSCW-Kompendium: Lehr- und Handbuch zum computerunterstützten kooperativen Arbeiten,,  Gerhard Schwabe, Norbert Streitz, Rainer Unland, Springer, Berlin; Auflage: 1 (1. August 2001)
Pol, M./Teunissen, R./Van, E. (2001), Software Testing: A Guide to the Tmap(r) Approach: A Guide to the TMap Approach,, Martin Pol, Ruud Teunissen, Erik Van Veenendaal, Verlag: Addison-Wesley Longman, Amsterdam; Auflage: illustrated edition (20. November 2001)
Booch, G./Jacobson, I./Rumbaugh, J. (2006), Das UML Benutzerhandbuch. Aktuell zur Version 2.0,,Grady Booch, James Rumbaugh, Ivar Jacobson, Verlag: Addison-Wesley, München (2006)
Gruhn, V./Pieper, D./Röttgers, C. (2006) MDA: Effektives Softwareengineering mit UML2 und Eclipse,, Volker Gruhn, Daniel Pieper, Carsten Röttgers, Verlag: Springer, Berlin; Auflage: 1 (Juli 2006)
Bremer, S./Oestereich, B. (2009) Analyse und Design mit UML 2.3: Objektorientierte Softwareentwicklung, ,Bernd Oestereich, Stefan Bremer, Verlag: Oldenbourg; Auflage: 9., aktualisierte und erweiterte Auflage. (15. September 2009)
Fitzgerald, J./Dennis A. (2009),Business Data Communications and Networking , Jerry Fitzgerald , Alan Dennis, Fitzgerald, Verlag: John Wiley \& Sons; Auflage: 10. (19. Januar 2009)
Dooley, K. (2001), Designing Large Scale LANs , Kevin Dooley, Verlag: O'Reilly Media; Auflage: 1 (November 2001)
IEEE(2011),Guidelines for use of the 24-bit Organizationally Unique Identifiers (OUI)
Burke, M./Kowalk, W. (1994), Rechnernetze, Wolfgang P. Kowalk , Manfred Burke, Verlag: Teubner Verlag; Auflage: 1 (1994)
Schreiner, R. (2009), Computernetzwerke. Von den Grundlagen zur Funktion und Anwendung, Rüdiger Schreiner, Hanser Fachbuch; Auflage: 3., überarb. Aufl. (21. April 2009)
Cisco (Hrsg.) (2004), Cisco Networking Academy Program 3. und 4. Semester: Autorisiertes Kursmaterial zur Bildungsinitiative Networking,, Cisco, , Verlag: Markt und Technik; Auflage: 3. A. 
Mauro, D./Schmidt, K. (2005),Essential SNMP: Help for System and Network Administrators,Verlag: O'Reilly Media; Auflage: 2nd ed. (23. September 2005)
vgl. Moos, A. (2004),Datenbank-Engineering: Analyse, Entwurf und Implementierung objektrelationaler Datenbanken - MIt UML, DB2-SQL und Java, Verlag: Vieweg+Teubner; Auflage: 3., überarb. u. erw. A. (30. März 2004)
Pernul, G./Unland R. (2003), S. 177, Datenbanken im Unternehmen: Analyse, Modellbildung und Einsatz [Taschenbuch],Verlag: Oldenbourg; Auflage: 2., korr. A. (26. Februar 2003)
King, T./Reese, G./Yarger, R. (2002), S. 137,MySQL. Einsatz und Programmierung,Verlag: O'Reilly; Auflage: 2. A. (Oktober 2002)
Dietrich, M./Ussar M. (2007), S. 98, Die Wissensdatenbank als Grundlage des Krisenmanagements,Verlag: Grin Verlag (November 2007)
Eickler, A./Kemper, A. (2006), S. 174f.,Datenbanksysteme. Eine Einführung,Verlag: Oldenbourg; Auflage: 6., aktualis. u. erw. A. (22. März 2006)
Preiß, N. (2007), Entwurf und Verarbeitung relationaler Datenbanken: Eine durchgängige und praxisorientierte Vorgehensweise, Verlag: Oldenbourg (1. April 2007)
Vossen, G. (2008), S. 27, Datenmodelle, Datenbanksprachen und Datenbankmanagementsysteme,Verlag: Oldenbourg; Auflage: 5. überarbeitete und erweiterte Auflage. (März 2008)
Heuer, A./Saake, G./Sattler, K. (2005), Datenbanken: Implementierungstechniken, Verlag: mitp; Auflage: 2., überarbeitete Auflage 2005 (13. Januar 2005)
Schicker, E. (2000), Datenbanken und SQL: Eine praxisorientierte Einführung mit Hinweisen zu Oracle und MS-Access Verlag: Teubner Verlag; Auflage: 3., durchges. A. (28. September 2000)
Fonth, E. (2007), S. 9, Multimedia-Datenbanken in Medienunternehmen: Technische Grundlagen, Verlag: Grin Verlag (November 2007)
Birnbam, D. (2003), S. 328,Microsoft Excel VBA: Professional Projects,Verlag: Premier Pr; Auflage: illustrated edition (August 2003)
Nicol, N./Albrecht, R. (2004), S. 963, Access 2003 programmieren. Professionelle Anwendungsentwicklung mit Access und VBA., Verlag: Addison Wesley Verlag (September 2004)
MacDonald, M. (2007), S. 5, Access 2007: The Missing Manual (Missing Manuals) Verlag: O'Reilly Media; Auflage: 1 (12. Januar 2007)
Zehoo, E. (2010), S. 2, Pro ODP.NET for Oracle Database 11g (Expert's Voice in Oracle), Verlag: Apress; Auflage: New. (31. Mai 2010)
Gilmore, J. (2010), S. 484, Beginning PHP and MySQL: From Novice to Professional (Expert's Voice in Web Development),Verlag: Apress; Auflage: New. (21. September 2010)
Cronan, J./Matthews, M. (2009), S. 240, Dynamic Web Programming: A Beginner's Guide, Verlag: Mcgraw-Hill Professional; Auflage: 1 (1. Dezember 2009)
Eisentraut, P./Helmle, B. (2010), PostgreSQL-Administration,Verlag: O'Reilly; Auflage: 2. Auflage. (22. Dezember 2010)
Apache Software Foundation (Hrsg.) (2010), Apache HTTP Server 2.2 Official Documentation - Volume I. Server Administration ,Verlag: Fultus Corp (2. April 2010)
vgl. Wutka, M. (2001), S. 76, J2EE Developer's Guide . JSP, Servlets, EJB 2.0, JNDI, JMS, JDBC, Corba, XML, RMI, Verlag: Markt+Technik (15. November 2001)
Taylor, A. (2007), S. 322, SQL für Dummies: Datenverwaltung vom Feinsten, Verlag: Wiley-VCH Verlag GmbH & Co. KGaA; Auflage: 4. aktualisierte Auflage (19. Januar 2007)
vgl. Roff, J. (2000), S. 3, ADO: ActiveX Data Objects, Verlag: O'Reilly Media (31. Dezember 2000)
%1Vgl. Bernd-Jrgen Falkowski(2002): Business Computing: Grundlagen und Standardsoftware, 1. Auflage, S.235
%1Vgl. Denis Hamann (2008): BELL, http://home.arcor.de/denis-hamann/profil/BELL.pdf S.30 f, Abruf 02.08.2009 16:50Uhr
%1Vgl. E. F. Codd(1970): A Relational Model of Data for Large Shared Data Banks in Commun. ACM, Vol 13, Nr. 6, S. 381
%1Vgl. Heinz Burnus(2007): Datenbankentwicklung in IT-Berufen, 1. Auflage, S.20
%1Vgl. Matthias Schubert(2007): Datenbanken, Theorie, Entwurf und Programmierung relationaler Datenbanken, 2. Auflage, S.293
%1Vgl. Microsoft: ASP Overview,http://www.asp.net/downloads/3.5-SP1/default.aspx, Abruf: 30. Juli 2009 8:56Uhr
%1Vgl. Microsoft: Get Started with IIS,http://www.iis.net/getstarted, Abruf: 30. Juli 2009 8:56Uhr
%1Vgl. Microsoft: IIS 7.0: bersicht ber die verfgbaren Features in IIS 7.0, http://msdn.microsoft.com/de-de/library/cc753198\%28WS.10\%29.aspx, Abruf: 01.August 16:04Uhr
%1Vgl. Peter Pin-Shan Chen(1976): The Entity-Relationship Model--Toward a Unified View of Data. In: ACM Transactions on Database Systems, Vol 1, No 1, S.10
%1Vgl. Peter Pin-Shan Chen(1976): The Entity-Relationship Model--Toward a Unified View of Data. In: ACM Transactions on Database Systems, Vol 1, No 1, S.19
%1Vgl. PostgreSQL: http://www.postgresql.org/about/licence, Abruf: 03.August 2009 15:27 Uhr
%1Vgl. Prof. Dr. Paul. Alpar(2001): Vorlesung, Datenorganisation und Datenbanken, Stand: 1.August 2009 12:51 Uhr, http://www.tekinci.de/skripte/DBDM/DB-SS2001.pdf
%1Vgl. Sun: JDBC-ODBC Bridge Driver, http://java.sun.com/j2se/1.3/docs/guide/jdbc/getstart/bridge.doc.html, Abruf: 30. Juli 2009 10:03 Uhr
%1Vgl. Tiobe Software(2009): TIOBE Programming Community Index for August 2009, http://www.tiobe.com/index.php/content/paperinfo/tpci/index.html, Abruf: 02. August 2009 14:31Uhr
%1Vgl. University of Maryland: How YouTube scales MySQL for its large databases, http://ebiquity.umbc.edu/blogger/2007/12/28/how-youtube-scales-mysql-for-its-large-databases/, Aufruf 1. August 2009 13:44Uhr
%1Vgl. Wikipedia, http://de.wikipedia.org/wiki/Wikipedia#Technik


%\end{thebibliography}
