\chapter*{Quellen/Literaturverzeichnis}
\addcontentsline{toc}{chapter}{Quellen/Literaturverzeichnis}

%\begin{thebibliography}{------}


%\bibitem[CODD1]{codd1}
%	,E. F. Codd.%
%    {\em A Relational Model of Data for Large Shared Data Banks}.
%	Commun. ACM, 1970, S. 377-387
Chen, P. (1976): The Entity-Relationship Model--Toward a Unified View of Data. In: ACM Transactions on Database Systems, Vol 1, No 1
Burnus, H. (2007): Datenbankentwicklung in IT-Berufen, 1. Auflage
Codd, E. F. (1970): A Relational Model of Data for Large Shared Data Banks in Commun. ACM, Vol 13, Nr. 6,
Alpar, P. (2001): Vorlesung, Datenorganisation und Datenbanken,  http://www.tekinci.de/skripte/DBDM/DB-SS2001.pdf,
Schubert, M. (2007): Datenbanken, Theorie, Entwurf und Programmierung relationaler Datenbanken, 2. Auflage
Falkowski, B. (2002): Business Computing: Grundlagen und Standardsoftware, 1. Auflage
Singh, S. (2009):Database Systems: Concepts, Design \& Applications, 
Tiobe Software, TIOBE Programming Community Index for January 2011
Schönsleben, P. (2007): Integrales Logistikmanagement. Planung und Steuerung der umfassenden Supply Chain, Paul Schönsleben, , Verlag: Springer, Berlin; Auflage: 4., überarb. u. erw. A. (März 2007)
Haas P./Johner, C. (2009), Praxishandbuch IT im Gesundheitswesen. Erfolgreich einführen, entwickeln, anwenden und betreiben, Christian Johner, , Peter Haas, Hanser Fachbuch (5. März 2009)
Schwabe, G./Streitz, N./Unland, R. (2001) CSCW-Kompendium: Lehr- und Handbuch zum computerunterstützten kooperativen Arbeiten,,  Gerhard Schwabe, Norbert Streitz, Rainer Unland, Springer, Berlin; Auflage: 1 (1. August 2001)
%1Vgl. Bernd-Jrgen Falkowski(2002): Business Computing: Grundlagen und Standardsoftware, 1. Auflage, S.235
%1Vgl. Denis Hamann (2008): BELL, http://home.arcor.de/denis-hamann/profil/BELL.pdf S.30 f, Abruf 02.08.2009 16:50Uhr
%1Vgl. E. F. Codd(1970): A Relational Model of Data for Large Shared Data Banks in Commun. ACM, Vol 13, Nr. 6, S. 381
%1Vgl. Heinz Burnus(2007): Datenbankentwicklung in IT-Berufen, 1. Auflage, S.20
%1Vgl. Matthias Schubert(2007): Datenbanken, Theorie, Entwurf und Programmierung relationaler Datenbanken, 2. Auflage, S.293
%1Vgl. Microsoft: ASP Overview,http://www.asp.net/downloads/3.5-SP1/default.aspx, Abruf: 30. Juli 2009 8:56Uhr
%1Vgl. Microsoft: Get Started with IIS,http://www.iis.net/getstarted, Abruf: 30. Juli 2009 8:56Uhr
%1Vgl. Microsoft: IIS 7.0: bersicht ber die verfgbaren Features in IIS 7.0, http://msdn.microsoft.com/de-de/library/cc753198\%28WS.10\%29.aspx, Abruf: 01.August 16:04Uhr
%1Vgl. Peter Pin-Shan Chen(1976): The Entity-Relationship Model--Toward a Unified View of Data. In: ACM Transactions on Database Systems, Vol 1, No 1, S.10
%1Vgl. Peter Pin-Shan Chen(1976): The Entity-Relationship Model--Toward a Unified View of Data. In: ACM Transactions on Database Systems, Vol 1, No 1, S.19
%1Vgl. PostgreSQL: http://www.postgresql.org/about/licence, Abruf: 03.August 2009 15:27 Uhr
%1Vgl. Prof. Dr. Paul. Alpar(2001): Vorlesung, Datenorganisation und Datenbanken, Stand: 1.August 2009 12:51 Uhr, http://www.tekinci.de/skripte/DBDM/DB-SS2001.pdf
%1Vgl. Sun: JDBC-ODBC Bridge Driver, http://java.sun.com/j2se/1.3/docs/guide/jdbc/getstart/bridge.doc.html, Abruf: 30. Juli 2009 10:03 Uhr
%1Vgl. Tiobe Software(2009): TIOBE Programming Community Index for August 2009, http://www.tiobe.com/index.php/content/paperinfo/tpci/index.html, Abruf: 02. August 2009 14:31Uhr
%1Vgl. University of Maryland: How YouTube scales MySQL for its large databases, http://ebiquity.umbc.edu/blogger/2007/12/28/how-youtube-scales-mysql-for-its-large-databases/, Aufruf 1. August 2009 13:44Uhr
%1Vgl. Wikipedia, http://de.wikipedia.org/wiki/Wikipedia#Technik


%\end{thebibliography}
