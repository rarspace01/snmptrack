% Anpassung des Seitenlayouts ----------------------------------------------
% 	siehe Seitenstil.tex
% --------------------------------------------------------------------------
\usepackage[
	automark,			% Kapitelangaben in Kopfzeile automatisch erstellen
	headsepline,	% Trennlinie unter Kopfzeile
	ilines				% Trennlinie linksbndig ausrichten
]{scrpage2}


%fuer die schriften
%\usepackage{mathptmx}
%\usepackage[scaled=.90]{helvet}
%\usepackage{courier}

% Anpassung an Landessprache -----------------------------------------------
% 	Verwendet globale Option german siehe \documentclass
% --------------------------------------------------------------------------
\usepackage{babel}

\usepackage{acronym}


% Grafiken -----------------------------------------------------------------
% 		Einbinden von Grafiken [draft oder final]
% 		Option [draft] bindet Bilder nicht ein - auch globale Option
% --------------------------------------------------------------------------
\usepackage[dvips,final]{graphicx}
\graphicspath{{Bilder/}} % Dort liegen die Bilder des Dokuments

% Befehle aus AMSTeX fr mathematische Symbole z.B. \boldsymbol \mathbb ----
\usepackage{amsmath,amsfonts}
%\usepackage{amsmath, amssymb, amstext, amsfonts, mathrsfs}


% Fr Index-Ausgabe; \printindex -------------------------------------------
\usepackage{makeidx}

% Einfache Definition der Zeilenabstnde und Seitenrnder etc. -------------
\usepackage{setspace}
\usepackage{geometry}


% Symbolverzeichnis --------------------------------------------------------
% 	Symbolverzeichnisse bequem erstellen, beruht auf MakeIndex.
% 		makeindex.exe %Name%.nlo -s nomencl.ist -o %Name%.nls
% 	erzeugt dann das Verzeichnis. Dieser Befehl kann z.B. im TeXnicCenter
%		als Postprozessor eingetragen werden, damit er nicht stndig manuell
%		ausgefhrt werden muss.
%		Die Definitionen sind ausgegliedert in die Datei Abkuerzungen.tex.
% --------------------------------------------------------------------------
\usepackage[intoc]{nomencl}
  \let\abbrev\nomenclature
  \renewcommand{\nomname}{Abkrzungsverzeichnis}
  \setlength{\nomlabelwidth}{.25\hsize}
  \renewcommand{\nomlabel}[1]{#1 \dotfill}
  \setlength{\nomitemsep}{-\parsep}


% Zum Umflieen von Bildern -------------------------------------------------
%\usepackage{floatflt}
\usepackage{float}


% Zum Einbinden von Programmcode --------------------------------------------
\usepackage{listings}
\usepackage{xcolor} 
\definecolor{hellgelb}{rgb}{1,1,0.9}
\definecolor{colKeys}{rgb}{0,0,1}
\definecolor{colIdentifier}{rgb}{0,0,0}
\definecolor{colComments}{rgb}{1,0,0}
\definecolor{colString}{rgb}{0,0.5,0}
\lstset{%
    float=hbp,%
    basicstyle=\texttt\small, %
    identifierstyle=\color{colIdentifier}, %
    keywordstyle=\color{colKeys}, %
    stringstyle=\color{colString}, %
    commentstyle=\color{colComments}, %
    columns=flexible, %
    tabsize=2, %
    frame=single, %
    extendedchars=true, %
    showspaces=false, %
    showstringspaces=false, %
    numbers=left, %
    numberstyle=\tiny, %
    breaklines=true, %
    backgroundcolor=\color{hellgelb}, %
    breakautoindent=true, %
%    captionpos=b%
}

% Lange URLs umbrechen etc. -------------------------------------------------
\usepackage{url}


% Wichtig fr korrekte Zitierweise ------------------------------------------
%\usepackage[square]{natbib}
%\includepackage[numeric]{natbib}

% Quellenangaben in eckige Klammern fassen ----------------------------------
%\bibpunct{[}{]}{;}{a}{}{,~}


% PDF-Optionen --------------------------------------------------------------
\usepackage[
bookmarks,
bookmarksopen=true,
pdftitle={\titel},
pdfauthor={\autor},
pdfcreator={\autor},
pdfsubject={\titel},
pdfkeywords={\titel},
colorlinks=true,
linkcolor=black, % einfache interne Verknpfungen
anchorcolor=black,% Ankertext
citecolor=black, % Verweise auf Literaturverzeichniseintrge im Text
filecolor=magenta, % Verknpfungen, die lokale Dateien ffnen
menucolor=black, % Acrobat-Menpunkte
urlcolor=black, 
% fr die Druckversion knnen die Farben ausgeschaltet werden:
%linkcolor=black, % einfache interne Verknpfungen
%anchorcolor=black,% Ankertext
%citecolor=black, % Verweise auf Literaturverzeichniseintrge im Text
%filecolor=black, % Verknpfungen, die lokale Dateien ffnen
%menucolor=black, % Acrobat-Menpunkte
%urlcolor=black, 
%backref, --test
%pagebackref,
plainpages=false,% zur korrekten Erstellung der Bookmarks
pdfpagelabels,% zur korrekten Erstellung der Bookmarks
hypertexnames=false,% zur korrekten Erstellung der Bookmarks
linktocpage % Seitenzahlen anstatt Text im Inhaltsverzeichnis verlinken
]{hyperref}

\pdfcompresslevel=1
%\pdfimageresolution=1200
%\pdfpkresolution=1200

% Zum fortlaufenden Durchnummerieren der Funoten ---------------------------
\usepackage{chngcntr}

\usepackage{caption}


% fr lange Tabellen
\usepackage{longtable}
\usepackage{array}
\usepackage{ragged2e}
\usepackage{lscape}

% Spaltendefinition rechtsbndig mit definierter Breite ---------------------
\newcolumntype{w}[1]{>{\raggedleft\hspace{0pt}}p{#1}}

% Formatierung von Listen ndern
\usepackage{paralist}
% Standardeinstellungen:
% \setdefaultleftmargin{2.5em}{2.2em}{1.87em}{1.7em}{1em}{1em}

%Zeilenabstand
\renewcommand{\baselinestretch}{1.5}\normalsize

%Schriftart
\renewcommand{\familydefault}{\sfdefault}