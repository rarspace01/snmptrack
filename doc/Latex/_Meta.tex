% Informationen ------------------------------------------------------------
% 	Definition von globalen Parametern, die im gesamten Dokument verwendet
% 	werden knnen (z.B auf dem Deckblatt etc.).
% --------------------------------------------------------------------------
\newcommand{\titel}{Systementwicklung eines Usertracking-Systems bei der Pirelli Deutschland GmbH}
\newcommand{\untertitel}{Leer}
\newcommand{\art}{Bachelorarbeit}
\newcommand{\fachgebiet}{Wirtschaftsinformatik}
\newcommand{\autor}{Denis Hamann}
\newcommand{\matrikelnr}{253977}
\newcommand{\studienbereich}{Wirtschaftsinformatik}
\newcommand{\kursbez}{WWI08B}
\newcommand{\firmenname}{Pirelli Deutschland GmbH}
\newcommand{\erstgutachter}{Olaf Rogge}
\newcommand{\zweitgutachter}{Gerd Hoffarth}
\newcommand{\jahr}{2011}

% Eigene Befehle
\newcommand{\todo}[1]{\textbf{\textsc{\textcolor{red}{(TODO: #1)}}}}

% Autorennamen in small caps
\newcommand{\AutorZ}[1]{\textsc{#1}}
\newcommand{\Autor}[1]{\AutorZ{\citeauthor{#1}}}

% Befehle zur semantischen Auszeichnung von Text
\newcommand{\NeuerBegriff}[1]{\textbf{#1}}
\newcommand{\Fachbegriff}[1]{\textit{#1}}
\newcommand{\Prozess}[1]{\textit{#1}}
\newcommand{\Webservice}[1]{\textit{#1}}
\newcommand{\Eingabe}[1]{\texttt{#1}}
\newcommand{\Code}[1]{\texttt{#1}}
\newcommand{\Datei}[1]{\texttt{#1}}
\newcommand{\Datentyp}[1]{\textsf{#1}}
\newcommand{\XMLElement}[1]{\textsf{#1}}

% Abkrzungen
\newcommand{\vgl}{Vgl.\ }
\newcommand{\ua}{\mbox{u.\,a.\ }}
\newcommand{\zB}{\mbox{z.\,B.\ }}
\newcommand{\bs}{$\backslash$}

