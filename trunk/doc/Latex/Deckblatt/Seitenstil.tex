% Zeilenabstand ------------------------------------------------------------
\onehalfspacing

%-- fu�notenabstand


% Seitenrnder -------------------------------------------------------------
\geometry{paper=a4paper,left=25mm,right=25mm,top=10mm, bottom=25mm}


% Kopf- und Fuzeilen ------------------------------------------------------
\pagestyle{scrheadings}

% Kopf- und Fuzeile auch auf Kapitelanfangsseiten -------------------------
\renewcommand*{\chapterpagestyle}{scrheadings}

% Schriftform der Kopfzeile ------------------------------------------------
\renewcommand{\headfont}{\normalfont}

% Kopfzeile ----------------------------------------------------------------
\ihead{\headmark} %\small{\untertitel}
\chead{\hspace{1mm}}
\ohead{Seite \thepage}
%\setlength{\headheight}{21mm} % Hhe der Kopfzeile
%\setheadwidth[0pt]{textwithmarginpar} % Kopfzeile ber den Text hinaus verbreitern
\setheadsepline[text]{0.4pt} % Trennlinie unter Kopfzeile

% Fuzeile -----------------------------------------------------------------
%\ifoot{\copyright\ \autor}
\cfoot{\hspace{1mm}}
\ofoot{\hspace{1mm}}

%Abbildungen
%Figures anpassung
\makeatletter
\@removefromreset{figure}{chapter}
\renewcommand*\thefigure{\@arabic\c@figure}
\makeatother

%%%
%\makeatletter
%\def\list@ftable{Tab. }\def\list@ffigure{Abbildung }
%\long\def\@caption#1[#2]#3{%
%  \par
%  \addcontentsline{\csname ext@#1\endcsname}{#1}%
%    {\csname list@f#1\endcsname\protect\numberline{%
%      \csname the#1\endcsname}{\ignorespaces #2}}%
%  \begingroup
%    \@parboxrestore
%    \if@minipage
%      \@setminipage
%    \fi
%    \normalsize
%    \@makecaption{\csname fnum@#1\endcsname}{\ignorespaces #3}\par
%  \endgroup}
%\makeatother
%%%
\DeclareCaptionListFormat{TESTLF}{\hspace{-19mm}Abbildung #1 #2}

\captionsetup{listformat=TESTLF}


% erzeugt ein wenig mehr Platz hinter einem Punkt --------------------------
\frenchspacing 

% Schusterjungen und Hurenkinder vermeiden
\clubpenalty = 10000
\widowpenalty = 10000 
\displaywidowpenalty = 10000


% Quellcode-Ausgabe formatieren --------------------------------------------
\lstset{numbers=left, numberstyle=\tiny, numbersep=5pt, breaklines=true}
\lstset{emph={square}, emphstyle=\color{red}, emph={[2]root,base}, emphstyle={[2]\color{blue}}}

% Funoten fortlaufend durchnummerieren ------------------------------------
\counterwithout{footnote}{chapter}
