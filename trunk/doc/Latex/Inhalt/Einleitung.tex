\chapter{Problemstellung und Zielsetzung der Arbeit}
\label{cha:Einleitung}

\section{Zielsetzung}

Der Verfasser war während der 5. Praxisphase in der IT-Abteilung im Bereich Infrastruktur der Pirelli Deutschland GmbH (PD) eingesetzt.
Zu den Aufgaben der Infrastruktur gehört die Sicherstellung des Betriebes der Anwendungssoftware.
Um den Einsatz dieser Software zu ermöglichen, kommt eine immense Anzahl an Hardware zum Einsatz.
Ein Teil der Hardware stellt unter anderem das Netzwerk dar.
Neben dem Aufbau und der Konfiguration des Netzwerkes zählt auch der Betrieb zu den Kernaufgaben.
Um den Betrieb des Netzwerks sicherzustellen, ist es notwendig, einen Überblick über dieses zu behalten.
Im Netzwerkmanagement gibt es hierfür einen speziellen Part, welcher sich unter dem Begriff des Netzwerkmonitoring zusammenfassen lässt.
Beim Netzwerkmonitoring gibt es die Möglichkeit sowohl Hardware als auch Anwendungen zu überwachen.
Ein spezieller Bereich konzentriert sich auf das Erfassen aller Geräte im Netzwerk.
Dieses wird z.B. von dem Hersteller Cisco als 'Usertracking' bezeichnet.
Hierfür soll in der Arbeit ein Entwurf erstellt werden, der auf Basis der Anforderungen des Unternehmens erstellt wird.
Um alle Anforderungen zu erfassen, soll im Voraus eine ausgiebige Analyse der Ist-Situation stattfinden.
An diese folgt dann die Implementierung des Systems.
Das Ziel soll es sein, Konzepte und Technologien aus der Literatur zu überprüfen und diese gezielt anzuwenden, um eine optimale Umsetzung zu erreichen. 
In der Arbeit selbst wird nicht im Detail auf die implementierte Lösung eingegangen, da dies den Umfang der Bachelorarbeit überschreitet.
Daher entfallen auch Erläuterungen zu eingesetzten Algorithmen und dem Quellcode.
Vielmehr wird sich auf den Entwurf und dessen Entscheidungsfindung konzentriert, da diese das Grundgerüst für die Implementierung liefern.
Im Mittelpunkt soll das Vorgehen, hilfreiche Konzepte aus der Theorie und eine kritische Betrachtung der Ausgangslage stehen, um eine optimale Umsetzung zu garantieren.\\


\section{Motivation}

Da bei PD, aufgrund der ausgelaufenenen Wartung, eine Aktualisierung der bestehenden 'CiscoWorks' Lösung ansteht, ist eine Untersuchung der aktuellen Situation notwendig.
Zunächst ist es von Bedeutung, die bestehenden Lizenzkosten der Software, zu senken.
Besonders im Zuge des immer weiter verbreiteten Einsatzes von OpenSource-Software im gewerblichen Bereich muss untersucht werden, ob es für die bisher eingesetzte Lösung ein Ersatz gibt bzw. ob eine eigene Lösung mit der Verwendung bereits existierender OpenSource-Software bewerkstelligt werden kann.
Neben den monetären Aspekten spielen aber auch technische Argumente eine Rolle.
So können bei eigenen Entwicklungen beispielsweise Anpassungen vorgenommen werden, die bei Standardsoftware nicht möglich sind.
Z.B. besteht dann die Möglichkeit, Switches von anderen Herstellern ebenfalls auszulesen, aber auch neuere Geräte ohne größere Verzögerung von dem System zu unterstützen.
Hinzukommt, dass durch das Auffinden der Hosts im Netzwerk die entsprechenden Switches bekannt sind, an die diese angeschlossen sind. Dadurch ergibt sich nicht nur die Möglichkeit einer Inventarisierung, sondern auch Diagnosemöglichkeiten. So kann für den Helpdesk die Möglichkeit bestehen, zu überprüfen, mit welcher Geschwindigkeit ein Host aktuell an das Netzwerk angeschlossen ist.
Dies ist vor allem sinnvoll, wenn die Verbindung zum jeweiligen Host sehr langsam ist, dann kann überprüft werden, ob es sich eventuell um ein Duplex-Mismatch oder ähnliches handelt.


\section{Vorgehensweise}

Die Arbeit gliedert sich in mehrere Abschnitte.
Zunächst wird auf wichtige Grundlagen eingegangen werden, welche für den Entwurf und die Implementierung eines passenden Systems notwendig sind.
Im Anschluss finde die Untersuchung der Ausgangssituation statt, sowie einer entsprechenden Anforderungsdefinition und Anforderungsanalyse.
Hier bei wird das bestehende System untersucht und anhand dessen in Zusammenarbeit der Mitarbeiter eine Anforderung für ein Ersatzsystem definiert.
Diese Anforderungen werden im Hinblick auf die technische und zeitliche Machbarkeit hin untersucht und mit bereits auf dem Markt existierenden Lösungen abgeglichen.
Im Anschluss folgt der, in der Softwareentwicklung typische, nächste Schritt, der Entwurf.
Durch diesen entsteht das Modell des späteren Systems auf Grundlage der Anforderungen mit Hilfe von Techniken aus der Theorie.
Daraufhin wird vom Verfasser eine Vielzahl von Untersuchungen vorgenommen, um die Designentscheidungen im Bezug auf deren Auswirkung auf die Umsetzung der Anforderungen sicherzustellen.
In diesem Zusammenhang wird auch auf den Zeitplan, sowie die Probleme bei der Implementierung eingegangen.
Zum Ende der Arbeit werden zusätzlich mögliche Anwendungsfelder der implementierten Lösung angesprochen und eine wirtschaftliche Betrachtung durchgeführt, welche überprüfen soll, ob die Implementierung des eigenen Systems eine ökonomisch sinnvolle Entscheidung ist.\\