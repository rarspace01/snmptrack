\chapter{Problemstellung und Zielsetzung der Arbeit}
\label{cha:Einleitung}

\section{Zielsetzung}

Der Verfasser war während der 5. Praxisphase in der IT-Abteilung im Bereich Infrastruktur der Pirelli Deutschland GmbH (PD) eingesetzt.
Zu den Aufgaben der Infrastruktur gehört die Sicherstellung des Betriebes der Anwendungssoftware.
Um den Einsatz dieser Software zu ermöglichen, kommt eine immense Anzahl an Hardware zum Einsatz.
Ein Teil der Hardware stellt unter anderem das Netzwerk dar.
Neben dem Aufbau und der Konfiguration des Netzwerkes zählt auch der Betrieb zu den Kernaufgaben.
Um den Betrieb des Netzwerks sicherzustellen, ist es notwendig, einen Überblick über dieses zu behalten.
Im Netzwerkmanagement gibt es hierfür einen speziellen Part, welcher sich unter dem Begriff des Netzwerkmonitoring zusammenfassen lässt.
Beim Netzwerkmonitoring gibt es die Möglichkeit sowohl Hardware als auch Anwendungen zu überwachen.
Ein spezieller Bereich konzentriert sich auf das Erfassen aller Geräte im Netzwerk.
Hierfür soll in der Arbeit ein Entwurf erstellt werden, der auf Basis der Anforderungen des Unternehmens erstellt wird.
Um alle Anforderungen zu erfassen, soll im Voraus eine ausgiebige Analyse der Ist-Situation stattfinden.
An diese folgt dann die Implementierung des Systems.
Das Ziel soll es sein, Konzepte und Technolgoien aus der Literatur zu überprüfen und diese gezielt anzuwenden, um eine optimale Umsetzung zu erreichen. 

Für dieses Umfeld existieren diverse Software, um diese Aufgabe zu bewerkstelligen.
Je nach Software können sowohl Hardware als auch Anwendungen kontrolliert werden.
Diese Arbeit konzentriert sich vor allem auf die Erfassung der Hardware in einem Netzwerk.
Aufgrund der Tatsache, dass bei PD eine Aktualisierung der kostenpflichtigen Lösung ‘CiscoWorks’ aus Kompatibilitätsgründen notwendig ist, soll aus Kostengründen ein Entwurf eines Systems zur Erfassung aller Geräte im Netzwerk (‘Usertracking’) entwickelt werden.
Es soll ein System entworfen werden, welches auf der Grundlage der Anforderungen des Unternehmens, sowie aufgrund eigener Untersuchungen, basiert.
In der Arbeit selbst wird nicht im Detail auf die implementierte Lösung eingegangen, da dies den Umfang der Bachelorarbeit überschreitet.
Daher entfallen auch Erläuterungen zu eingesetzten Algorithmen und dem Quellcode.
Vielmehr wird sich auf den Entwurf und dessen Entscheidungsfindung konzentriert, da diese das Grundgerüst für die Implementierung liefern.
Im Mittelpunkt soll das Vorgehen, hilfreiche Konzepte aus der Theorie und eine kritische Betrachtung der Ausgangslage stehen, um eine optimale Umsetzung zu garantieren.\\

\section{Motivation}

Zunächst ist es von Bedeutung, die bestehenden Lizenzkosten der Software, zu senken.
Besonders im Zuge des immer weiter verbreiteten Einsatzes von OpenSource-Software im gewerblichen Bereich muss untersucht werden, ob es für die bisher eingesetzte Lösung ein Ersatz gibt bzw. ob eine eigene Lösung mit der Verwendung bereits existierender OpenSource-Software bewerkstelligt werden kann.
Neben den monetären Aspekten spielen aber auch technische Argumente eine Rolle.
So können bei eigenen Entwicklungen beispielsweise Anpassungen vorgenommen werden, die bei Standardsoftware nicht möglich sind.
Z.B. besteht dann die Möglichkeit, Switches von anderen Herstellern ebenfalls auszulesen, aber auch neuere Geräte ohne größere Verzögerung von dem System zu unterstützen.

\section{Vorgehensweise}

Hierzu soll zunächst auf wichtige Grundlagen eingegangen werden, welche zu einer Implementierung notwendig sind.
Im Anschluss soll eine Untersuchung der Ausgangslage erfolgen und anhand dieser die entsprechenden Anforderungen abgeleitet und diese wiederum kritisch untersucht werden.
Im Anschluss werden bereits existierende Lösungen untersucht und deren Eigenschaften erläutert.
Danach wird die Entscheidung für eine eigene Implementierung eines Systems getroffen.
Um dies bewerkestelligen zu können, wird ein Entwurf eines solchen Systems erstellt.
Daraufhin werden eine Vielzahl von Untersuchungen vorgenommen, um die Designentscheidungen im Bezug auf deren Auswirkung auf die Umsetzung der Anforderungen sicherzustellen.
In diesem Zusammenhang wird auch auf den Zeitplan, sowie die Probleme bei der implementierung eingegangen.
Zum Ende der Arbeit werden zusätzlich mögliche Anwendungsfelder der Implementierten Lösung angesprochen und eine wirtschaftliche Betrachtung durchgeführt, welche überprüfen soll, ob die Implementierung eines eigenen Softwaresystems eine ökonomisch sinnvolle Entscheidung wäre.\\