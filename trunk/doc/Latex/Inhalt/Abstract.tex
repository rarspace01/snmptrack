\section*{Abstract}
\label{sec:Abstract}

\begin{tabbing}
Verfasser:	 \hspace{20mm} \= \autor\\
Kurs:	  \>\kursbez\\
Unternehmen:	  \>\firmenname\\
Thema:	  \> Systementwicklung eines Usertracking-Systems\\
		\> bei der Pirelli Deutschland GmbH\\
\end{tabbing}

In der vorliegenden Arbeit wird ein System entwickelt, welches zum Erfassen aller im Netzwerk befindlichen Geräte dient.
Hierfür wird zunächst auf die grundlegenden Technologien und Werkzeuge eingegangen, die für eine Umsetzung des Systems notwendig sind und in der Literatur behandelt werden.
Neben Methoden zur Softwareentwicklung, wird auch auf die technischen Aspekte der Kommunikation im Netzwerk, sowie auf Datenbanken eingegangen.
Im anschließenden Teil wird die aktuelle Situation bei Pirelli Deutschland GmbH untersucht. Anhand dieser werden Anforderungen an das System in Absprache mit den späteren Benutzern formuliert und diese in Hinblick auf die Realisierbarkeit überprüft.
Im Haupteil wird der Entwurf für das System erstellt. Dies erfolgt anhand der Modelle aus der UML.
Bevor dieser Entwurf umgesetzt wird, finden mehrere Untersuchungen statt, die Ansätze aus der Theorie überprüfen. Diese Ansätze sind für die optimale Umsetzung des Projektes vorteilhaft und werden daher näher untersucht.
Gegen Ende wird auf die Probleme, die bei der Implementierung auftraten, und deren Lösung eingegangen. 
Zum Abschluss der Arbeit werden weitere Anwendungsmöglichkeiten erörtert, sowie auf die Wirtschaftlichkeit der Umsetzung eingegangen.