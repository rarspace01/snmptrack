\chapter*{Quellen/Literaturverzeichnis}
\addcontentsline{toc}{chapter}{Quellen/Literaturverzeichnis}

%\begin{thebibliography}{------}


%\bibitem[CODD1]{codd1}
%	,E. F. Codd.%
%    {\em A Relational Model of Data for Large Shared Data Banks}.
%	Commun. ACM, 1970, S. 377-387
\textbf{Alpar, P. (2001),} Vorlesung, Datenorganisation und Datenbanken,\\http://www.tekinci.de/skripte/DBDM/DB-SS2001.pdf - Aufruf der Seite: 07.02.2011\\\\
\textbf{Apache Software Foundation (Hrsg.) (2010),} Apache HTTP Server 2.2 Official Documentation - Volume I. Server Administration, Kalifornien, USA 2010\\\\
\textbf{Barth, W. (2008),} Nagios: System and Network Monitoring, 2. Auflage, Kalifornien, USA 2008\\\\
\textbf{Birnbam, D. (2003),} Microsoft Excel VBA: Professional Projects, Massachusetts, USA 2003\\\\
\textbf{Booch, G./Jacobson, I./Rumbaugh, J. (2006),} Das UML Benutzerhandbuch. Aktuell zur Version 2.0, München 2006\\\\
\textbf{Brandt-Pook, H./Kollmeier, R. (2008),} Softwareentwicklung kompakt und verständlich: Wie Softwaresysteme entstehen, Wiesbaden 2008\\\\
\textbf{Bremer, S./Oestereich, B. (2009),} Analyse und Design mit UML 2.3: Objektorientierte Softwareentwicklung, 9. Auflage, München 2009\\\\
\textbf{Burke, M./Kowalk, W. (1994),} Rechnernetze, Wiesbaden 1994\\\\
\textbf{Burnus, H. (2007),} Datenbankentwicklung in IT-Berufen: Eine praktisch orientierte Einführung mit MS Access und MySQL, Wiesbaden 2007\\\\
\textbf{Chen, P. (1976),} The Entity-Relationship Model--Toward a Unified View of Data. In: ACM Transactions on Database Systems, Vol 1, No 1, New York, USA 1976\\\\
\clearpage
\textbf{Cisco (Hrsg.) (2004),} Cisco Networking Academy Program 3. und 4. Semester: Autorisiertes Kursmaterial zur Bildungsinitiative Networking, 3. Auflage, München 2004\\\\
\textbf{Cisco (Hrsg.) (2005),} How to Collect CPU Utilization on Cisco IOS Devices Using SNMP, Kalifornien, USA 2005\\\\
\textbf{Cockburn, A. (2007),} Use Cases effektiv erstellen, Frechen 2007)\\\\
\textbf{Codd, E. F. (1970),} A Relational Model of Data for Large Shared Data Banks in Commun. In: ACM, Vol 13, No 6, New York, USA\\\\
\textbf{Cronan, J./Matthews, M. (2009),} Dynamic Web Programming: A Beginner's Guide, New York, USA 2009\\\\
\textbf{Desikan, S./ Ramesh, G. (2007),} Software Testing: Principles and Practice, London, Vereinigtes Königreich 2007\\\\
\textbf{Dietrich, M./Ussar M. (2007),} Die Wissensdatenbank als Grundlage des Krisenmanagements, München 2007\\\\
\textbf{Dooley, K. (2001),} Designing Large Scale LANs , Kevin Dooley, Kalifornien, USA 2001\\\\
\textbf{Eickler, A./Kemper, A. (2006),} Datenbanksysteme. Eine Einführung, 6. Auflage, München 2006\\\\
\textbf{Eisentraut, P./Helmle, B. (2010),} PostgreSQL-Administration, 2. Auflage, Kalifornien, USA 2010\\\\
\textbf{Falkowski, B. (2002),} Business Computing: Grundlagen und Standardsoftware, Berlin 2002\\\\
\textbf{Fitzgerald, J./Dennis A. (2009),} Business Data Communications and Networking, New Jersey, USA 2009\\\\
\textbf{Fonth, E. (2007),} Multimedia-Datenbanken in Medienunternehmen: Technische Grundlagen, München 2007\\\\
\textbf{Gilmore, J. (2010),} Beginning PHP and MySQL: From Novice to Professional (Expert's Voice in Web Development), New York, USA 2010\\\\
\textbf{Gruhn, V./Pieper, D./Röttgers, C. (2006),} MDA: Effektives Softwareengineering mit UML2 und Eclipse, Berlin 2006\\\\
\textbf{GULP (Hrsg.) (2011),} GULP-Stundesatz Kalkulator,\\http://www.gulp.de/cgi-gulp/trendneu.exe/MONEYFORMDLL\\ - Aufruf der Seite: 07.02.2011\\\\
\textbf{Haas P./Johner, C. (2009),} Praxishandbuch IT im Gesundheitswesen. Erfolgreich einführen, entwickeln, anwenden und betreiben, München 2009\\\\
\textbf{Heuer, A./Saake, G./Sattler, K. (2005),} Datenbanken: Implementierungstechniken, Auflage: 2., Frechen 2005\\\\
\textbf{Hoffarth, G. (2010), Anforderungsdefinition}
\textbf{IEEE (Hrsg.) (2011),} Guidelines for use of the 24-bit Organizationally Unique Identifiers (OUI), Kalifornien, USA 2011\\\\
\textbf{IEEE (Hrsg.) (2011),} OUIs,\\http://standards.ieee.org/develop/regauth/oui/public.html - Aufruf der Seite: 07.02.2011\\\\
\textbf{King, T./Reese, G./Yarger, R. (2002),} MySQL. Einsatz und Programmierung, Kalifornien, USA 2002\\\\
\textbf{Kolbenschlag, W./Reiner, W./Teich, I (2007),} S. 171, Der richtige Weg zur Softwareauswahl: Lastenheft, Pflichtenheft, Compliance, Erfolgskontrolle, Berlin 2007\\\\
\textbf{Ljaci, N. (2010),} Integration von MockUp-Konzepten in die Spezifikation grafischer Bedienoberflächen, München 2010\\\\
\clearpage
\textbf{MacDonald, M. (2007),} Access 2007: The Missing Manual, Kalifornien, USA 2007\\\\
\textbf{Mauro, D./Schmidt, K. (2005),} Essential SNMP: Help for System and Network Administrators, 2. Auflage, Kalifornien, USA 2005\\\\
\textbf{Microsoft (Hrsg.) (2011),} IIS 7.0: Übersicht über die verfügbaren Features in IIS 7.0,\\http://msdn.microsoft.com/de-de/library/cc753198\%28WS.10\%29.aspx - Aufruf der Seite: 06.02.2011\\\\
\textbf{Mindfactory (2011),} Preisliste,\\http://www.mindfactory.de/product\_info.php/info/p700694\\ - Aufruf der Seite: 06.02.2011\\\\
\textbf{Moos, A. (2004),} Datenbank-Engineering: Analyse, Entwurf und Implementierung objektrelationaler Datenbanken - Mit UML, DB2-SQL und Java, 3. Auflage, Wiesbaden 2004\\\\
\textbf{Müller, J. (2004),} Tirith V 1.2 Datenbank Schema, Darmstadt 2004\\\\ 
\textbf{Nicol, N./Albrecht, R. (2004),} Access 2003 programmieren. Professionelle Anwendungsentwicklung mit Access und VBA., München 2004\\\\
\textbf{Olbrich, A. (2003),} Netze - Protokolle - Spezifikationen. Die Grundlagen für die erfolgreiche Praxis, Wiesbaden 2003\\\\
\textbf{Oracle (Hrsg.) (2011)}, Database Reference,\\http://download.oracle.com/docs/cd/B19306\_01/server.102/b14237/initparams138.htm - Aufruf der Seite: 07.02.2011\\\\
\textbf{Oracle (Hrsg.) (2011),} JDBC-ODBC Bridge Driver,\\ http://download.oracle.com/javase/1.3/docs/guide/jdbc/getstart/bridge.doc.html  - Aufruf der Seite: 06.02.2011\\\\
\clearpage
\textbf{Oracle (Hrsg.) (2011),} Wiki, http://wiki.oracle.com/page/OPEN\_CURSORS\\ - Aufruf der Seite: 07.02.2011\\\\
\textbf{Pernul, G./Unland R. (2003),} Datenbanken im Unternehmen: Analyse, Modellbildung und Einsatz, 2. Auflage, München 2003\\\\
\textbf{Pilone, D. (2003),} UML. Kurz und gut., Köln 2003\\\\
\textbf{Pol, M./Teunissen, R./Van, E. (2001),} Software Testing: A Guide to the Tmap(r) Approach: A Guide to the TMap Approach,, Martin Pol, Ruud Teunissen, Erik Van Veenendaal, Amsterdam 2001\\\\
\textbf{PostgreSQL (2011),} Lizenz, http://www.postgresql.org/about/licence - Aufruf der Seite: 06.02.2011\\\\
\textbf{Preiß, N. (2007),} Entwurf und Verarbeitung relationaler Datenbanken: Eine durchgängige und praxisorientierte Vorgehensweise, München 2007\\\\
\textbf{Roff, J. (2000),} ADO: ActiveX Data Objects, Kalifornien, USA 2000\\\\
\textbf{Schicker, E. (2000),} Datenbanken und SQL: Eine praxisorientierte Einführung mit Hinweisen zu Oracle und MS-Access, 3. Auflage, Wiesbaden 2000\\\\
\textbf{Schönsleben, P. (2007),} Integrales Logistikmanagement. Planung und Steuerung der umfassenden Supply Chain, Paul Schönsleben, Berlin 2007\\\\
\textbf{Schreiner, R. (2009),} Computernetzwerke. Von den Grundlagen zur Funktion und Anwendung, 3. Auflage, München 2009\\\\
\textbf{Schubert, M. (2007),} Datenbanken, Theorie, Entwurf und Programmierung relationaler Datenbanken, 2. Auflage, Wiesbaden 2007\\\\
\textbf{Schwabe, G./Streitz, N./Unland, R. (2001),} CSCW-Kompendium: Lehr- und Handbuch zum computerunterstützten kooperativen Arbeiten, Berlin 2001\\\\
\textbf{Singh, S. (2009),} Database Systems: Concepts, Design \& Applications, New Jersey 2009\\\\
\textbf{Statistisches Bundesamt (Hrsg.) (2011),} Genesis Datenbank: Verbraucherpreisindex Jahr 1991-2011, Wiesbaden\\\\
\textbf{Taylor, A. (2007),} SQL für Dummies: Datenverwaltung vom Feinsten, 4. Auflage, Weinheim 2007\\\\
\textbf{Tiobe Software (Hrsg.),} TIOBE Programming Community Index for January 2011, Eindhoven, Niederlande 2011\\\\
\textbf{Unhelkar, B. (2005),} Practical Object Oriented Design, New York,USA 2005\\\\
\textbf{Volkwein, G. (2007),} Konzept zur effizienten Bereitstellung von Steuerungsfunktionalität für die NC-Simulation, München 2007\\\\
\textbf{Vossen, G. (2008),}, Datenmodelle, Datenbanksprachen und Datenbankmanagementsysteme, 5. Auflage, München 2008\\\\
\textbf{Wutka, M. (2001),} J2EE Developer's Guide . JSP, Servlets, EJB 2.0, JNDI, JMS, JDBC, Corba, XML, RMI, München 2001\\\\
\textbf{Zehoo, E. (2010),} Pro ODP.NET for Oracle Database 11g (Expert's Voice in Oracle), New York, USA 2010\\\\

%1Vgl. Bernd-Jrgen Falkowski(2002): Business Computing: Grundlagen und Standardsoftware, 1. Auflage, S.235
%1Vgl. Denis Hamann (2008): BELL, http://home.arcor.de/denis-hamann/profil/BELL.pdf S.30 f, Abruf 02.08.2009 16:50Uhr
%1Vgl. E. F. Codd(1970): A Relational Model of Data for Large Shared Data Banks in Commun. ACM, Vol 13, Nr. 6, S. 381
%1Vgl. Heinz Burnus(2007): Datenbankentwicklung in IT-Berufen, 1. Auflage, S.20
%1Vgl. Matthias Schubert(2007): Datenbanken, Theorie, Entwurf und Programmierung relationaler Datenbanken, 2. Auflage, S.293
%1Vgl. Microsoft: ASP Overview,http://www.asp.net/downloads/3.5-SP1/default.aspx, Abruf: 30. Juli 2009 8:56Uhr
%1Vgl. Microsoft: Get Started with IIS,http://www.iis.net/getstarted, Abruf: 30. Juli 2009 8:56Uhr
%1Vgl. Microsoft: IIS 7.0: bersicht ber die verfgbaren Features in IIS 7.0, http://msdn.microsoft.com/de-de/library/cc753198\%28WS.10\%29.aspx, Abruf: 01.August 16:04Uhr
%1Vgl. Peter Pin-Shan Chen(1976): The Entity-Relationship Model--Toward a Unified View of Data. In: ACM Transactions on Database Systems, Vol 1, No 1, S.10
%1Vgl. Peter Pin-Shan Chen(1976): The Entity-Relationship Model--Toward a Unified View of Data. In: ACM Transactions on Database Systems, Vol 1, No 1, S.19
%1Vgl. PostgreSQL: http://www.postgresql.org/about/licence, Abruf: 03.August 2009 15:27 Uhr
%1Vgl. Prof. Dr. Paul. Alpar(2001): Vorlesung, Datenorganisation und Datenbanken, Stand: 1.August 2009 12:51 Uhr, http://www.tekinci.de/skripte/DBDM/DB-SS2001.pdf
%1Vgl. Sun: JDBC-ODBC Bridge Driver, http://java.sun.com/j2se/1.3/docs/guide/jdbc/getstart/bridge.doc.html, Abruf: 30. Juli 2009 10:03 Uhr
%1Vgl. Tiobe Software(2009): TIOBE Programming Community Index for August 2009, http://www.tiobe.com/index.php/content/paperinfo/tpci/index.html, Abruf: 02. August 2009 14:31Uhr
%1Vgl. University of Maryland: How YouTube scales MySQL for its large databases, http://ebiquity.umbc.edu/blogger/2007/12/28/how-youtube-scales-mysql-for-its-large-databases/, Aufruf 1. August 2009 13:44Uhr
%1Vgl. Wikipedia, http://de.wikipedia.org/wiki/Wikipedia#Technik


%\end{thebibliography}
